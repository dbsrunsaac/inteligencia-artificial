\documentclass[conference]{IEEEtran}
\IEEEoverridecommandlockouts
% The preceding line is only needed to identify funding in the first footnote. If that is unneeded, please comment it out.
\usepackage{cite}
\usepackage{amsmath,amssymb,amsfonts}
\usepackage{algorithmic}
\usepackage{graphicx}
\usepackage{textcomp}
\usepackage{xcolor}
\usepackage{tabularx}
\usepackage{multirow}
\usepackage{graphics} % for pdf, bitmapped graphics files
\usepackage{subfig}
\usepackage{subcaption}
\usepackage{hyperref}
\usepackage{academicons}
\usepackage{xcolor}
\usepackage{listings}
\usepackage{tabularx} % Asegúrate de incluir este paquete
\usepackage{float}
\usepackage{tikz}
\usetikzlibrary{shapes.geometric, arrows}

\usetikzlibrary{shapes.geometric, arrows}

\tikzstyle{startstop} = [rectangle, rounded corners, minimum width=3cm, minimum height=1cm,text centered, draw=black, fill=red!30]
\tikzstyle{process} = [rectangle, minimum width=3cm, minimum height=1cm, text centered, draw=black, fill=blue!30]
\tikzstyle{arrow} = [thick,->,>=stealth]


\def\BibTeX{{\rm B\kern-.05em{\sc i\kern-.025em b}\kern-.08em
		T\kern-.1667em\lower.7ex\hbox{E}\kern-.125emX}}

% Color Enlace
\definecolor{colorEnlace}{RGB}{0, 0, 0}
\hypersetup{
	colorlinks=true,
	linkcolor=colorEnlace,
	citecolor=colorEnlace,
	urlcolor=colorEnlace,
	pdfauthor={Davis Bremdow Salazar Roa},
	pdftitle={Control Fuzzy - Faja de un motor}
}

\definecolor{mybg}{rgb}{0.97,0.97,0.97}
\definecolor{mygray}{gray}{0.4}
\definecolor{mygreen}{rgb}{0,0.6,0}
\definecolor{myblue}{rgb}{0,0,0.8}
\definecolor{mypurple}{rgb}{0.58,0,0.82}
\definecolor{myred}{rgb}{0.7,0,0}

\lstdefinelanguage{MatlabEnhanced}{
	language=Matlab,
	morekeywords={[2]linspace,plot,title,xlabel,ylabel,legend,grid},
	morekeywords={[3]sin,cos,exp,log,sqrt},
	keywordstyle=\color{myblue}\bfseries,
	keywordstyle=[2]\color{mypurple},
	keywordstyle=[3]\color{myred},
	commentstyle=\color{mygreen}\itshape,
	stringstyle=\color{mygray},
	morecomment=[l]%
}

\lstset{
	language=MatlabEnhanced,
	backgroundcolor=\color{mybg},
	frame=single,
	basicstyle=\ttfamily\small,
	showstringspaces=false,
	numbers=none,              %
	xleftmargin=0pt,           %
	framexleftmargin=0pt,      
	framexrightmargin=0pt,
	framextopmargin=2pt,
	framexbottommargin=2pt,
	breaklines=true,
	tabsize=1,
}

% Control 
\usepackage{amsmath}
\begin{document}
	
	\title{CONTROL FUZZY - EXAMEN PARCIAL}
	\author{
		\makebox[\textwidth][c]{\large\textbf{Universidad Nacional de San Antonio Abad del Cusco}}\\
		\makebox[\textwidth][c]{\normalsize\textit{Escuela profesional de Ingeniería Electrónica}}\\
		\makebox[\textwidth][c]{\normalsize\textit{Inteligencia Artificial}}\\
		\and
		\IEEEauthorblockN{Ing. Ruben Dario Florez Zela}
		\IEEEauthorblockA{ Ingeniero Electrónico \\
			Cusco, Perú \\
			ruben.florez@unsaac.edu.pe}
		\and
		\IEEEauthorblockN{Davis Bremdow Salazar Roa}
		\IEEEauthorblockA{Estudiante de Ingeniería Electrónica \\
			Cusco, Perú \\
			200353@unsaac.edu.pe}
	}
	
	\maketitle
	
	\begin{abstract}
		
		
	\end{abstract}
	
	\begin{IEEEkeywords}
		Control Fuzzy, Lógica difusa, MATLAB, FuzzyLogicDesigner, Mamdani
	\end{IEEEkeywords}
	
	\section{Simulación MATLAB}
	
	Para el sistema se definen 2 entradas, una salida y para las cuales corresponde las siguientes funciones de pertenencia que se muestran en \ref{fig:per-e}, \ref{fig:per-de}
	
	\begin{figure}[H]
		\centering
		\includegraphics[width=0.4\textwidth]{media/per-e}
		\caption{Función de pertenencia - entrada e}
		\label{fig:per-e}
	\end{figure}
	
	\begin{figure}[H]
		\centering
		\includegraphics[width=0.4\textwidth]{media/per-de}
		\caption{Función de pertenencia - entrada de}
		\label{fig:per-de}
	\end{figure}
	
	Y para la salida en la figura \ref{fig:per-p}
	
	\begin{figure}[H]
		\centering
		\includegraphics[width=0.4\textwidth]{media/per-p}
		\caption{Función de pertenencia - salida P}
		\label{fig:per-p}
	\end{figure}
	
	
	Para el sistema de control de temperatura mediante lógica fuzzy se define el siguiente conjunto de bloques en simulink \ref{fig:bloques-simulink}, \textbf{además para cada caso la defuzzificación se muestran en las figuras mediante el tool box de MATLAB} en \ref{fig:ede-55-3} y \ref{fig:ede-63-3} respectivamente.
	
	\begin{figure}[H]
		\centering
		\includegraphics[width=0.5\textwidth]{media/bloques-simulink}
		\caption{Diagrama de bloques simulink}
		\label{fig:bloques-simulink}
	\end{figure}
	
	Luego para las entradas de 10, 35 y 50 se tiene como salida se tienen las figuras \ref{fig:in-10}, \ref{fig:in-35}, \ref{fig:in-50} respectivamente.
	
	\begin{figure}[H]
		\centering
		\includegraphics[width=0.4\textwidth]{media/in-10}
		\caption{Respuesta - Temperatura para un set point de 10}
		\label{fig:in-10}
	\end{figure}
	\begin{figure}[H]
		\centering
		\includegraphics[width=0.4\textwidth]{media/in-35}
		\caption{Respuesta - Temperatura para un set point de 35}
		\label{fig:in-35}
	\end{figure}
	\begin{figure}[H]
		\centering
		\includegraphics[width=0.4\textwidth]{media/in-50}
		\caption{Respuesta - Temperatura para un set point de 50}
		\label{fig:in-50}
	\end{figure}
	
	
	Por otro lado en el toolbox de Fuzzy se puede apreciar las salidas cuando se tienen los valores de e y de para : (55, 3) y (-63, -3) en las figuras \ref{fig:ede-55-3} y \ref{fig:ede-63-3} respectivamente.
	
	\begin{figure}[H]
		\centering
		\includegraphics[width=0.5\textwidth]{media/ede-55-3}
		\caption{Valor de P para e = 55 y de = 3}
		\label{fig:ede-55-3}
	\end{figure}
	
	\begin{figure}[H]
		\centering
		\includegraphics[width=0.5\textwidth]{media/ede-63-3}
		\caption{Valor de P para e = -63 y de = -3}
		\label{fig:ede-63-3}
	\end{figure}
	
	
	Para e = 55 y de = 3, se tiene como salida para el valor de P = 2.25 y para un e = -63 y de = -3 el valor de es equivalente a -2.25
	
	\newpage
	
	\section{Simulación en Python}
	
	Por otro en la simulación en python se tiene el sistema equivalente en MATLAB para lo cual se definieron sus funciones de pertenencia para las entradas y salidas que se muestran en las figuras \ref{fig:py-per-e}, \ref{fig:py-per-de} y \ref{fig:py-per-p} respectivamente.
	
	\begin{figure}[H]
		\centering
		\includegraphics[width=0.5\textwidth]{media/py-per-e}
		\caption{Funciones de pertenencia para - entrada e}
		\label{fig:py-per-e}
	\end{figure}
	
	
	\begin{figure}[H]
		\centering
		\includegraphics[width=0.5\textwidth]{media/py-per-de}
		\caption{Funciones de pertenencia para - entrada de}
		\label{fig:py-per-de}
	\end{figure}
	
	\begin{figure}[H]
		\centering
		\includegraphics[width=0.5\textwidth]{media/py-per-p}
		\caption{Funciones de pertenencia para - salida P}
		\label{fig:py-per-p}
	\end{figure}
	
	Y para las salidas para los valores propuestos previamente en MATLAB se tiene que: para e = 55 y de = 3, se tiene como salida para el valor de P = 2.2436, como se muestra en la figura \ref{fig:py-out-55-3} 
	
	\begin{figure}[H]
		\centering
		\includegraphics[width=0.5\textwidth]{media/py-out-55-3}
		\caption{Respuesta para una e = 55 y de = 3}
		\label{fig:py-out-55-3}
	\end{figure}
	
	Y para un e = -63 y de = -3 el valor de es equivalente a -2.2499 como se muestra en la figura \ref{fig:py-out-63-3}
	
	\begin{figure}[H]
		\centering
		\includegraphics[width=0.5\textwidth]{media/py-out-63-3}
		\caption{Respuesta para una e = -63 y de = -3}
		\label{fig:py-out-63-3}
	\end{figure}
	
	
	\newpage
	\newpage
	
	\section{Preguntas teóricas}
	
	\section{P.2}
	
	El trade off es el balance entre el overfiting y el underfitign que a su vez se relaciona directamente con el sesgos y bias respectivamente que implica que el modelo se sobre ajusta y no comprende los datos completamente.
	
	La complejidad del modelo puede hacer que los datos se sobre ajusten o que no aprendan nada es por tanto que analizar los datos y escoger el algoritmo de entrenamiento es de vital importancia para lograr un tradeoff adecuado.
	
	\section{P. 3}
	
	Supervisado = datos etiquetados
	No supervisado = datos sin supervisar
	
	Y los objetivos de forma generar es el tipo de modelo que se podrá entrenar para el caso del modelo supervisado se tienen algoritmos de regresión y clasificación mientras que con los no supervisados se cuentan con aplicaciones de agrupación de datos y reconocimiento de patrones ocultos.
	
	Las metricas en regresión no son adecuadas para clasificación el tipo de dato de salida que se pueda obtener en el caso de la clasificación son discretos y en la regresión continuos.
	
	\section{P. 5}
	Accuracy: Exactitud
	
	
	\bibliographystyle{IEEEtran}
	\bibliography{biblio}
	
\end{document}